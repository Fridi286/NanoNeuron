\documentclass[
  a4paper,            % DIN A4
  DIV=10,             % Schriftgröße und Satzspiegel
  oneside,            % einseitiger Druck
  BCOR=5mm,           % Bindungskorrektur
  parskip=half,       % Halber Abstand zwischen Absätzen
  numbers=noenddot,   % Kein Punkt hinter Kapitelnummern
  bibliography=totoc,  % Literaturverzeichnis im Inhaltsverzeichnis
  listof=totoc        % Abbildungs- und Tabellenverzeichnis im Inhaltsverzeichnis
]{scrartcl}
\usepackage{../style/termpaperstyle}

%\usepackage{layout}       % Layout Debugging
%\usepackage{showframe}    % Layout Debugging
\usepackage{lipsum}       % for example only
\usepackage{blindtext}    % for example only

\sisetup{locale = DE}     % siunitx locale setup
%\DeclareSIUnit \fps{fps}  % a custom unit (usage: \SI{24}{\fps})

\begin{document}
% !TEX root = ../termpaper.tex
%
% configurations
%

% English Language support
% -> uncomment if needed
% Beta!
%\fullenglish{yes}
\fullenglish{no}

% text field
%-> replace supervisor names with correct ones
\firstSupervisor{Prof. Dr. Peer Stelldinger}
\secondSupervisor{}    % only if needed, otherwise left blank

% text field
%-> replace title with your title of the seminar work
\termPaperTitle{Analyse des Einflusses von Trainingsstrategie, Aktivierungsfunktion und Netztiefe auf Konvergenz und Generalisierung eines selbst implementierten neuronalen Netzes}
\termPaperTitleEN{}

% text field
%-> replace the key words with your own key words or remove the words
\keywordsDE{Überwachtes Lernen, Gradientenabstieg, Mini-Batch, ReLU, Sigmoid, Netztiefe, MNIST, Konvergenz}
\keywordsEN{}

% text field
%-> replace john with your name
\termPaperAuthor{Frithjof Beims}

% text field
%-> enter the submission date
\submissionDate{21.02.2026}

% switch - uncomment only one
%-> uncomment NDA or public
%\NDA{yes}
\NDA{no}

% switch - uncomment only one
%-> uncomment cover or cover Corporate Design 2017
%\Cover{CD2017}
%\Cover{CD2017NoLogo}
\Cover{Std2018}

% switch - uncomment only one
%-> uncomment the kind of seminar you are in
%\termpaperKind{E}            % Exposé in bachelor course
%\termpaperKind{S}            % seminar in bachelor course
\termpaperKind{H}            % home assignment in Bachelor course
%\termpaperKind{Project}      % Project report
%\termpaperKind{Pro}          % pro-seminar in bachelor course
%\termpaperKind{GSem}         % foundation seminar in master computer science
%\termpaperKind{HSem}         % main seminar in master computer science
%\termpaperKind{MH}           % home assignment in Master course
%\termpaperKind{Research1}    % foundation research workshop (Forschungswerkstatt 1) in master computer science
%\termpaperKind{Research2}    % research workshop (Forschungswerkstatt 2) in master computer science

% switch - uncomment only one
%-> uncomment the study course you are in
%\studycourse{INF_ITS}   % Informatik Technischer Systeme
\studycourse{INF_AI}   % Angewandte Informatik
%\studycourse{INF_WI}   % Wirtschaftsinformatik
%\studycourse{INF_ECS}  % European Computer Science
%\studycourse{EMI_EI}   % Elektrotechnik und Informationstechnik
%\studycourse{EMI_REE}  % Regenerative Energiesysteme und Energiemanagement
%\studycourse{EMI_IE}   % Inforamation Engineering
%\studycourse{BMT}      % Mechatronik
%\studycourse{INF_MaI}      % Master Informatik
%\studycourse{EMI_MaA}      % Master Automatisierung
%\studycourse{EMI_MaICE}    % Master Information and Communication Engieering
%\studycourse{EMI_MaMS}     % Master Microelectronic Systems
    % load all settings

%\layout{}                 % Layout Debugging

\hyphenation{Ba-che-lor-the-sis Mas-ter-the-sis}

% Cover page here, no page number
\ICoverPage

% PDF Metadata
\input{../style/metadata}

% Titlepage is page one even if the number is not shown.
\pagenumbering{roman}

% Table of contents here
\newpage
\tableofcontents

% Uncomment if list of source code is needed (rarely).
%\lstlistoflistings  % requires package listings, needs to uncommenting of usepackage

% path to the chapters folder is set to find the images used there
\graphicspath{ {./chapters/} }

% Chapters
\clearpage
\pagenumbering{arabic}

\begin{abstract}
\end{abstract}

\ITextBlockKeywords

\section{Einleitung}

Diese Arbeit analysiert das Trainingsverhalten von vollständig verbundenen neuronalen Netzen anhand des MNIST-Zahlenklassifikationsproblems. Die Zielsetzung ist eine Analyse des Einflusses ausgewählter architektonischer und trainingsbezogener Faktoren auf Konvergenz und Generalisierung.

Um die zugrunde liegenden Mechanismen besser nachvollziehen zu können, wird ein Vollvermaschtes Neuronales Netz selbstständig implementiert. Auf diese Weise können der Gradientenfluss, der Verlauf des Verlusts und die Stabilität der Lernkurven genau untersucht werden. Als standardisierter Referenzdatensatz ermöglicht MNIST reproduzierbare und vergleichbare Experimente.

\subsection{Zielsetzung der Arbeit}

Diese Arbeit hat das Ziel, den Einfluss von vier wesentlichen Faktoren auf das Konvergenz- und Generalisierungsverhalten eines neuronalen Netzes zu analysieren:

\begin{itemize}
    \item die Wahl der Aktivierungsfunktion (Sigmoid vs. ReLU),
    \item die Trainingsstrategie (Online Learning vs. Mini-Batch Training),
    \item die Netztiefe (ein Hidden Layer vs. drei Hidden Layer),
    \item sowie die Wahl von geeigneten Hyperparametern (Learning Rate, Anzahl der Neuronen, Batch-Size)
\end{itemize}

Dabei soll analysiert werden, wie sich diese Faktoren auf Trainingsgenauigkeit, Testgenauigkeit, Verlustverlauf und Stabilität der Lernkurven und die Generalisierung auswirken. Ein besonderer Fokus liegt auf der Frage, inwiefern ReLU das Vanishing-Gradient-Problem mildert und ob Mini-Batch-Training zu stabileren Optimierungsverläufen führt und ein schnelleres Training ermöglicht.

\subsection{Formale Problemdefinition}

Formal wird das Problem als überwachtes Klassifikationsproblem definiert. Gegeben sei eine Trainingsmenge 
\[
\mathcal{D} = \{(x_i, y_i)\}_{i=1}^{N},
\]
wobei $x_i \in \mathbb{R}^{784}$ einen normierten Eingabevektor (28×28 Pixel) und $y_i \in \{0, \dots, 9\}$ die zugehörige Klassenbezeichnung (im Folgenden: Label) repräsentiert. \cite{MNISTDataset}

Das Modell sei ein vollständig verbundenes neuronales Netz mit einer variablen Anzahl versteckter Schichten. Jede Schicht besteht aus einer affinen Transformation $z = W x + b$ gefolgt von einer nichtlinearen Aktivierungsfunktion.
Der Parameterraum umfasst sämtliche Gewichtsmatrizen und Bias-Vektoren aller Schichten.

Als Verlustfunktion wird die Cross-Entropy-Loss verwendet. 
Die versteckten Schichten nutzen entweder die Sigmoid- oder die ReLU-Aktivierungsfunktion, 
während im Ausgabelayer eine Softmax-Funktion zur Modellierung der Klassenwahrscheinlichkeiten eingesetzt wird.

Das zu lösende Optimierungsproblem lautet:

\[
\min_{\theta} \; \mathcal{L}(\theta) = 
\frac{1}{N} \sum_{i=1}^{N} 
\ell(f_{\theta}(x_i), y_i),
\]

\cite{2014-Shai-Shalev-Shwartz-and-Shai-Ben-David}

wobei $\theta$ alle Modellparameter bezeichnet und $\ell$ die Cross-Entropy-Loss-Funktion darstellt. 
Die Optimierung erfolgt mittels Gradientenabstieg und Backpropagation.


\subsection{Leitfrage der Untersuchung}

Die zentrale Leitfrage dieser Arbeit lautet:

\textit{Wie beeinflussen Aktivierungsfunktion, Trainingsstrategie und Netztiefe das Konvergenzverhalten und die Generalisierungsfähigkeit vollständig verbundener neuronaler Netze auf dem MNIST-Klassifikationsproblem?}

Darüber hinaus soll untersucht werden, ob sich systematische Interaktionseffekte zwischen diesen Faktoren erkennen lassen und welche Konfiguration unter identischen Hyperparametern das stabilste und effizienteste Lernverhalten zeigt.

% !TEX root = ../termpaper.tex
% @author Frithjof Beims
%

\section{Theoretische Grundlagen}

\subsection{Überwachtes Lernen}

Beim überwachten Lernen wird ein Modell auf Basis gelabelter Trainingsdaten optimiert. 
Gegeben ist eine Trainingsmenge

\[
\mathcal{D} = \{(x_i, y_i)\}_{i=1}^{N},
\]

wobei $x_i \in \mathbb{R}^d$ einen Eingabevektor und $y_i$ die zugehörige Klassenbezeichnung darstellt. 
Ziel ist es, eine Funktion $f_\theta(x)$ mit Parametern $\theta$ zu bestimmen, 
die neue Eingaben korrekt klassifiziert. \cite{Deeplearningchapter1}

Im Fall des MNIST-Datensatzes handelt es sich um ein Mehrklassenproblem mit zehn Klassen (Ziffern 0–9).

\subsection{Feedforward-Neuronale Netze} 

Ein Fully Connected Neural Network besteht aus mehreren hintereinandergeschalteten Schichten.  
Jede Schicht berechnet zunächst eine affine Transformation \cite{Deeplearningchapter1} \cite{Nahua2017Kang}

\[
z^{(l)} = W^{(l)} a^{(l-1)} + b^{(l)},
\]

gefolgt von einer nichtlinearen Aktivierungsfunktion (ReLU oder Sigmoid). Die Nichtlinearität ist notwendig, da mehrere rein lineare Transformationen wieder einer einzigen linearen Funktion entsprechen würden.

\subsection{Aktivierungsfunktionen}

In dieser Arbeit werden zwei Aktivierungsfunktionen untersucht.

\subsubsection{Sigmoid}

Die Sigmoid-Funktion ist definiert als

\[
\sigma(x) = \frac{1}{1 + e^{-x}}.
\]

Sie bildet Eingaben auf das Intervall $(0,1)$ ab. \cite{Deeplearningchapter1}
Ein Nachteil besteht darin, dass für große Beträge von $x$ die Ableitung sehr klein wird, 
was zu langsamer oder stagnierender Gradientenweitergabe führen kann (Vanishing Gradient).

\subsubsection{ReLU}

Die Rectified Linear Unit ist definiert als

\[
\text{ReLU}(x) = \max(0, x).
\]

Sie besitzt für positive Eingaben eine konstante Ableitung von 1 
und reduziert dadurch das Risiko verschwindender Gradienten in tiefen Netzen. \cite{Sigmoid-vs-ReLU}

\subsection{Softmax und Mehrklassenklassifikation}

Für die Mehrklassenklassifikation wird im Ausgabelayer eine Softmax-Funktion verwendet:

\[
\text{Softmax}(z_i) = 
\frac{e^{z_i}}{\sum_{j=1}^{K} e^{z_j}}.
\]

Sie transformiert die Ausgabewerte in eine Wahrscheinlichkeitsverteilung über die $K$ Klassen. \cite{Einfuehrung-in-Neuronale-Netze}

\subsection{Verlustfunktionen}

Zur Optimierung wird der Cross-Entropy-Loss verwendet:

\[
\mathcal{L} = - \sum_{i=1}^{K} y_i \log(\hat{y}_i).
\]

Dabei bezeichnet $y_i$ das One-Hot-Label und $\hat{y}_i$ die vorhergesagte Klassenwahrscheinlichkeit. \cite{Neural-Networks-Part-6:-Cross-Entropy}

\subsection{Gradientenabstieg und Backpropagation}

Die Modellparameter werden mittels Gradientenabstieg aktualisiert:

\[
\theta \leftarrow \theta - \eta \nabla_\theta \mathcal{L},
\]

wobei $\eta$ die Lernrate bezeichnet. \cite{DeepLearningChapter2}

Die Berechnung der Gradienten erfolgt effizient durch Backpropagation, bei der die Ableitungen der Verlustfunktion schichtweise unter Anwendung der Kettenregel rückwärts durch das Netzwerk propagiert werden. \cite{DeepLearningChapter3} \cite{DeepLearningChapter4}

\subsection{Online- und Mini-Batch-Training}

Beim Online Learning erfolgt die Aktualisierung der Gewichte nach jedem einzelnen Trainingsbeispiel. 
Dies führt zu hoher Gradientenvarianz, kann jedoch schnelle Anpassungen ermöglichen.

Beim Mini-Batch-Training werden mehrere Trainingsbeispiele gemeinsam verarbeitet. 
Die Gradienten werden über ein Batch gemittelt, wodurch stabilere Updates entstehen und die Varianz reduziert wird. \cite{Zhang2023}

% !TEX root = ../termpaper.tex
% @author Frithjof Beims
%

\section{Methodik}

\subsection{Implementierungsdetails}

Das neuronale Netz wurde in Python implementiert. Als numerisches Backend wird NumPy verwendet, optional auch CuPy für GPU-Betrieb. Die Implementierung umfasst Forward-Pass, Backpropagation, Online-Training sowie Mini-Batch-Training.

Die Gewichtsmatrizen werden zu Beginn zufällig aus einer Gleichverteilung im Intervall $[-0.1, 0.1]$ initialisiert, 
während die Bias-Vektoren mit Null initialisiert werden. 
Zur Sicherstellung der Reproduzierbarkeit wird ein Seed gesetzt.

Die Eingabedaten werden auf den Bereich $[0,1]$ normalisiert, indem die Pixelwerte durch 255 dividiert werden. 
Die Klassenlabels werden in One-Hot-Vektoren transformiert.

Je nach Konfiguration werden unterschiedliche Aktivierungs- und Verlustfunktionen verwendet:

\begin{itemize}
    \item Sigmoid in allen Schichten mit quadratischer Fehlerfunktion (MSE) im sampleweisen Training,
    \item ReLU in den Hidden Layern mit Softmax im Output-Layer und Cross-Entropy-Loss.
\end{itemize}

\subsection{Untersuchte Netzarchitekturen}

Untersucht werden Fully Connected Netzwerke mit unterschiedlicher Tiefe und Breite. 
Die Anzahl der Hidden Layer variiert zwischen ein- und mehrschichtigen Architekturen. 
Die Anzahl der Neuronen pro Schicht wird systematisch variiert.

Die automatisiert getesteten Architekturen umfassen:

\begin{itemize}
    \item Ein Hidden Layer mit 128, 256 oder 512 Neuronen,
    \item Zwei Hidden Layer mit [32,32], [64,64] oder [128,64],
    \item Drei bis vier Hidden Layer mit [256,128,64] bzw. [256,128,64,32].
\end{itemize}

Der Output-Layer besitzt stets zehn Neuronen entsprechend der zehn MNIST-Klassen, beziehungsweise der Zahlen von 0-9.

\subsection{Trainingsstrategie}

Es werden zwei Trainingsstrategien betrachtet:

\begin{itemize}
    \item Online Learning (Batchgröße = 1),
    \item Mini-Batch-Training mit Batchgröße 32.
\end{itemize}

Beim Mini-Batch-Training werden die Trainingsdaten zu Beginn jeder Epoche zufällig permutiert. 
Die Gradienten werden über das jeweilige Batch gemittelt und anschließend zur Parameteraktualisierung verwendet.

Die Parameter werden mittels klassischem Gradientenabstieg aktualisiert:

\[
\theta \leftarrow \theta - \eta \nabla_\theta \mathcal{L},
\]

wobei $\eta$ die Lernrate bezeichnet.

\subsection{Hyperparameter}

Im automatisierten Parallel-Experiment werden folgende Hyperparameter systematisch variiert:

\begin{itemize}
    \item Lernrate $\eta \in \{0.0001, 0.001, 0.01, 0.025, 0.05, 0.1\}$,
    \item Batchgröße $=32$ (bei Mini-Batch-Training),
    \item Anzahl der Hidden Layer und Neuronenzahl pro Layer,
    \item Seed $=42$.
\end{itemize}

Die Anzahl der Epochen beträgt in den gezeigten Vergleichsläufen 50.

\subsection{Evaluationskriterien}

Die Modellleistung wird anhand folgender Kennzahlen bewertet:

\begin{itemize}
    \item Trainingsgenauigkeit,
    \item Testgenauigkeit,
    \item Test-Loss,
    \item Generalisierungsgap (Differenz zwischen Trainings- und Testgenauigkeit),
    \item Konvergenzverlauf über die Epochen.
\end{itemize}

Die Testdaten werden nicht zur Parameteranpassung verwendet, sondern ausschließlich zur Bewertung der Generalisierungsfähigkeit des Modells.

\subsection{Versuchsdesign}

Zur systematischen Analyse der Einflussfaktoren werden mehrere Konfigurationen automatisiert trainiert. Die Trainingsläufe werden parallelisiert ausgeführt, wobei jede Konfiguration unabhängig trainiert und ihre Metriken fortlaufend gespeichert werden.

Alle Experimente verwenden identische Trainings- und Testdatensplits. Nicht untersuchte Hyperparameter werden konstant gehalten, um die Effekte von Trainingsstrategie und Architektur möglichst isoliert zu betrachten.

\section{Experimente und Ergebnisse}

\subsection{Gesamtüberblick}

Es wurden verschiedene Kombinationen aus Aktivierungsfunktion,
Netztiefe, Lernrate und Trainingsstrategie untersucht.
Die höchste Validierungsgenauigkeit wurde mit einem
ReLU-Netz mit Mini-Batch-Training erreicht
(HL-[256,128,64], LR=0.1, Batchgröße 32).
Dieses Modell erzielte eine Validierungsgenauigkeit von 0.9847
in Epoche 17.

\begin{figure}
    \centering
    \includegraphics[width=0.8\linewidth]{image.png}
    \caption{Best Model: (HL-[256,128,64], LR=0.1, Batchgröße 32)}
    \label{fig:placeholder}
\end{figure}

Mehrere leistungsstarke Modelle erreichen ihre maximale
Validierungsgenauigkeit zwischen Epoche 15 und 30.
Anschließend bleiben die Werte weitgehend stabil.

\subsection{Einfluss der Aktivierungsfunktion}

ReLU-Modelle erreichen in allen getesteten Konfigurationen
höhere Validierungsgenauigkeiten als Sigmoid-Modelle.

Die beste ReLU-Konfiguration erzielt 98.47\%,
während die beste Sigmoid-Konfiguration 97.66\% erreicht.

Darüber hinaus erreichen ReLU-Modelle ihre maximale
Validierungsgenauigkeit früher im Trainingsverlauf.

\begin{figure}
    \centering
    \includegraphics[width=0.8\linewidth]{best_models_by_category.png}
    \caption{Best Models by Category}
    \label{fig:placeholder}
\end{figure}

\subsection{Einfluss der Trainingsstrategie}

Der Vergleich zwischen Online Learning und Mini-Batch-Training
zeigt Unterschiede in Genauigkeit und Stabilität.

Für ReLU-Modelle steigt die maximale Validierungsgenauigkeit
von 97.65\% (ohne Batch) auf 98.47\% (mit Batch).

Batch-Modelle zeigen im Verlauf der Epochen
gleichmäßigere Lernkurven
als Modelle ohne Batch-Verarbeitung.

\subsection{Einfluss der Netztiefe}

Architekturen mit mehreren Hidden Layern
erzielen leicht höhere Validierungsgenauigkeiten
als Modelle mit nur einem Hidden Layer.

Ein einzelner Hidden Layer mit 256 Neuronen
erreicht 98.34\%,
während die dreischichtige Architektur
(256-128-64) 98.47\% erreicht.

Der Unterschied zwischen den Architekturen
liegt im Bereich weniger Zehntelprozentpunkte.

\subsection{Konvergenzverhalten}

Die meisten leistungsstarken Modelle
erreichen ihre maximale Validierungsgenauigkeit
vor Epoche 30.
In späteren Epochen treten nur noch geringe Veränderungen auf.

Sigmoid-Modelle zeigen insgesamt
langsamere Verbesserungen über die Epochen hinweg.

\subsection{Analyse der Trainingszeit}

Zusätzlich wurde die Trainingszeit erfasst.

Mini-Batch-Training benötigt im Durchschnitt
5.26 Sekunden pro Epoche,
während Online Learning im Mittel
8.06 Sekunden pro Epoche benötigt.

ReLU-Modelle benötigen durchschnittlich
5.72 Sekunden pro Epoche,
Sigmoid-Modelle 7.82 Sekunden.

Kleinere Netze mit 16 Neuronen im Hidden Layer
erreichen Gesamttrainingszeiten von etwa 60 Sekunden
für 50 Epochen,
erzielen jedoch geringere Validierungsgenauigkeiten
als größere Architekturen.

Viele Modelle erreichen ihre beste
Validierungsgenauigkeit deutlich vor dem Ende
des Trainingslaufs.

\begin{figure}
    \centering
    \includegraphics[width=0.8\linewidth]{Trainingszeit.png}
    \caption{Traingingtime per Epoch by Trainmode}
    \label{fig:placeholder}
\end{figure}

\subsection{Zusammenfassung der Ergebnisse}

Die experimentellen Ergebnisse zeigen:

\begin{itemize}
    \item ReLU-Modelle erzielen höhere Validierungsgenauigkeiten als Sigmoid-Modelle.
    \item Mini-Batch-Training führt zu besseren Ergebnissen als Online Learning.
    \item Zusätzliche Tiefe erhöht die Genauigkeit leicht.
    \item Batch-Modelle sind schneller als Modelle ohne Batch-Verarbeitung.
\end{itemize}

% !TEX root = ../termpaper.tex
% @author Frithjof Beims
%

\section{Diskussion}

\subsection{Interpretation der Konvergenzverläufe}

Die Experimente zeigen deutliche Unterschiede im Konvergenzverhalten
in Abhängigkeit von Aktivierungsfunktion, Trainingsstrategie und Netztiefe.

ReLU-Modelle mit Mini-Batch-Training erreichen ihre maximale Validierungsgenauigkeit bereits nach etwa 15–25 Epochen, während Sigmoid-Modelle insgesamt langsamer konvergieren
und flachere Verbesserungsverläufe aufweisen. Die frühe Stabilisierung der leistungsstarken Modelle deutet darauf hin, dass die Verlustfunktion effektiv minimiert wird und sich das Optimierungsverfahren in einem stabilen Bereich befindet.

Dieses Verhalten ist konsistent mit der theoretischen Analyse des Gradientenflusses.
Die Sigmoid-Funktion besitzt in gesättigten Bereichen sehr kleine Ableitungen, wodurch Gradienten insbesondere in tieferen Netzen abgeschwächt werden können. ReLU vermeidet dieses Problem für positive Aktivierungen und ermöglicht dadurch stabilere Parameterupdates.

\subsection{Rolle der Netztiefe}

Die Ergebnisse zeigen, dass zusätzliche Tiefe nur einen moderaten Leistungszuwachs bringt. Der Unterschied zwischen einem einzelnen Hidden Layer und einer dreischichtigen Architektur liegt im Bereich weniger Zehntelprozentpunkte.

Dies deutet darauf hin, dass MNIST keine stark hierarchische Merkmalsrepräsentation erfordert. Fully Connected Netze mit ausreichender Breite scheinen bereits genügend Repräsentationskapazität zu besitzen, um die zugrunde liegende Struktur der Daten zu modellieren.

Gleichzeitig wird deutlich, dass nicht allein die Repräsentationskapazität
über die Leistungsfähigkeit entscheidet, sondern maßgeblich die Optimierbarkeit der Parameter. Eine größere Modellkomplexität erhöht die Schwierigkeit der Verlustminimierung, insbesondere bei ungünstigem Gradientenverhalten. Die Kombination aus Tiefe und Sigmoid führt zu langsameren Konvergenzverläufen, während ReLU die Optimierbarkeit deutlich verbessert.

\subsection{Mini-Batch-Training als Stabilisierungselement}

Mini-Batch-Training zeigt in den Experimenten eine klare Überlegenheit gegenüber reinem Online Learning.

Die Mittelung der Gradienten über mehrere Beispiele reduziert die Varianz der Updates. Dies führt zu stabileren Lernkurven und konsistenteren Konvergenzverläufen. Neben der verbesserten Stabilität zeigt sich auch eine leicht höhere Generalisierungsleistung.

Aus praktischer Perspektive ist zudem relevant, dass Mini-Batch-Training rechnerisch effizienter ist. Die vektorisierte Verarbeitung mehrerer Beispiele ermöglicht eine bessere Nutzung der zugrunde liegenden Matrixoperationen. Damit erweist sich Mini-Batch-Training sowohl aus Optimierungs- als auch aus Effizienzperspektive als die geeignetere Trainingsstrategie.

\subsection{Generalisierung und Bias-Varianz-Kompromiss}

Die Differenz zwischen Trainings- und Testgenauigkeit bleibt bei den leistungsstärksten Modellen gering. Ein ausgeprägtes Overfitting ist nicht zu beobachten.

Die geringe Generalisierungslücke deutet darauf hin, dass sich die untersuchten Modelle in einem günstigen Bereich des Bias-Varianz-Kompromisses befinden. Die gewählte Modellkomplexität ist ausreichend, um die Trainingsdaten gut zu approximieren, ohne eine übermäßige Varianz gegenüber neuen Daten zu entwickeln.

Dass zusätzliche Tiefe nur geringe Leistungsgewinne bringt, spricht ebenfalls dafür, dass die grundlegende Komplexität des Problems bereits mit moderater Modellkapazität abgedeckt werden kann.

\subsection{Grenzen des Ansatzes}

Trotz der konsistenten Ergebnisse besitzt der gewählte Ansatz mehrere Einschränkungen.

Erstens wird ausschließlich ein Fully Connected Netzwerk verwendet. Die räumliche Struktur der Bilddaten wird nicht explizit berücksichtigt. Convolutional Neural Networks würden hier eine strukturangepasste Merkmalsextraktion ermöglichen.

Zweitens werden keine expliziten Regularisierungsverfahren eingesetzt. Techniken wie Dropout oder Gewichtsnormierung könnten insbesondere bei größeren Architekturen zusätzliche Stabilität bringen.

Drittens erfolgt die Optimierung mittels klassischem Gradientenabstieg ohne adaptive Verfahren. Moderne Optimierer wie Adam oder RMSprop könnten insbesondere bei tieferen Netzen eine schnellere und robustere Konvergenz ermöglichen.

Schließlich ist MNIST ein vergleichsweise einfacher Benchmark. Die Übertragbarkeit der beobachteten Effekte auf komplexere Datensätze muss daher kritisch betrachtet werden.

\subsection{Einordnung und Beantwortung der Leitfrage}

Die zentrale Fragestellung dieser Arbeit lautete, wie sich Aktivierungsfunktion, Trainingsstrategie und Netztiefe auf Konvergenzverhalten und Generalisierung vollständig verbundener neuronaler Netze auswirken.

Die empirischen Ergebnisse zeigen, dass Aktivierungsfunktion und Trainingsstrategie einen deutlich stärkeren Einfluss besitzen als die reine Netztiefe. ReLU verbessert die Optimierbarkeit tiefer Netze, während Mini-Batch-Training zu stabileren und effizienteren Lernverläufen führt. Zusätzliche Tiefe erhöht die Modellkapazität, liefert jedoch bei MNIST nur begrenzten Mehrwert.

Außerdem zeigt die Analyse der Trainingszeiten, dass Mini-Batch-Training nicht nur stabilere Lernverläufe ermöglicht, sondern auch zu einer geringeren Trainingsdauer führt. Die Kombination aus ReLU und Mini-Batch erweist sich daher sowohl hinsichtlich Optimierbarkeit als auch Effizienz als die insgesamt leistungsfähigste Konfiguration.

Damit bestätigt die Arbeit unter kontrollierten experimentellen Bedingungen wesentliche theoretische Annahmen über Gradientenfluss, Varianzreduktion und Repräsentationskapazität.
% !TEX root = ../termpaper.tex
% @author Frithjof Beims
%

\section{Fazit und Ausblick}

Ziel dieser Arbeit war es, den Einfluss von Aktivierungsfunktion,
Trainingsstrategie und Netztiefe auf das Konvergenzverhalten
und die Generalisierungsfähigkeit vollständig verbundener
neuronaler Netze am Beispiel des MNIST-Datensatzes zu untersuchen.

Die Ergebnisse zeigen, dass nicht primär die Modelltiefe,
sondern vor allem die Optimierungsdynamik
über die Leistungsfähigkeit entscheidet.
Insbesondere die Kombination aus ReLU-Aktivierung
und Mini-Batch-Training erweist sich als robust,
stabil und effizient.
Zusätzliche Tiefe erhöht zwar die Repräsentationskapazität,
führt jedoch bei diesem Datensatz nur zu begrenzten Verbesserungen.

Die Arbeit verdeutlicht damit,
dass architektonische Komplexität allein
keinen entscheidenden Vorteil bietet,
wenn Optimierbarkeit und Trainingsstrategie
nicht angemessen berücksichtigt werden.
Für das betrachtete Problem
stellt ein moderat tiefes,
gut optimierbares Netzwerk
eine zweckmäßige Lösung dar.

Zukünftige Untersuchungen könnten den Ansatz
auf struktursensitive Architekturen wie
Convolutional Neural Networks übertragen,
adaptive Optimierungsverfahren einbeziehen
oder komplexere Datensätze analysieren,
um die Generalisierbarkeit der beobachteten Effekte
weiter zu prüfen.
% Add additional chapters here

%\bibliographystyle{plain}
\nocite{*}
\bibliographystyle{dinat}
\bibliography{literature}

\begin{abstract}
\end{abstract}

\addsec{Erklärung zum Einsatz von KI-Werkzeugen}

Bei der Erstellung der Hausarbeit habe ich KI-Tools in folgendem Umfang verwendet:

Für die Implementierung in Pyhton wurde hauptsächlich Copilot (Claude 4.5) verwendet. Das Tool unterstützte bei der Erstellung einer grafischen Oberfläche zur Zahlenerkennung, beim Refaktoring und Formatieren des Codes, bei der Erweiterung des Single-Prozessor Trainingsskript auf Multiprozessor-Berechnung, sowie bei der Fehlersuche.
Außerdem unterstütze Copilot bei der Erstellung der Strukturierung eines automatisierten Auswertungsskripts über mehrere Tausend-Epochen-Trainingsdaten mit verschiedenen Netzkonfigurationen.

Bei der Ausarbeitung der schriftlichen Arbeit wurde ChatGPT unterstützend eingesetzt, insbesondere zur Formatierung mathematischer Formeln im LaTeX-Format, zur sprachlichen Überarbeitung einzelner Textpassagen, zur Strukturierung der Gliederung sowie zur Unterstützung bei der Literaturrecherche. Zudem wurde ChatGPT punktuell zur Klärung fachlicher Verständnisfragen herangezogen. Die inhaltliche Bewertung und Einordnung der erhaltenen Informationen erfolgte eigenständig.

Die inhaltliche Konzeption der Arbeit, die Entwicklung und Implementierung des neuronalen Netzes, die Durchführung der Experimente sowie die wissenschaftliche Interpretation der Ergebnisse erfolgten eigenständig.

\Istatement

\end{document}
